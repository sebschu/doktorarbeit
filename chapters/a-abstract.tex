Speakers exhibit considerable production variability at all levels of linguistic representations. This raises the question how successful communication is nevertheless possible most of the time.
In this dissertation, I investigate this question and I study to what extent listeners adapt to variable use words using the example of uncertainty expressions such as \textit{might} and \textit{probably}.  In several web-based experiments, I show that listeners exhibit uncertainty in their expectations about a generic speaker's use of uncertainty expressions;  that listeners update production expectations to match a specific speaker's use of uncertainty expressions after a brief exposure to that speaker; and that updated production expectations result in updated speaker-specific interpretations of uncertainty expressions. 

I further investigate the associated cognitive processes and I investigate what kind of representations listeners update during semantic-pragmatic adaptation. To this end, I  present a novel Bayesian computational model of production expectations of uncertainty expressions and a novel model of the adaptation process based on Bayesian belief updating.  Through a series of simulations, I find that post-adaptation behavior is best predicted by a model that assumes that listeners update both speaker-specific semantic representations and speaker-specific utterance choice preferences, suggesting that listeners update at least these two types of representations as a result of adaptation. 

Finally, I show in additional experiments that listeners adapt to multiple speakers and that adaptation behavior is modulated by non-linguistic contextual factors such as the speaker's mood. 

This work has implications for both semantic theories of uncertainty expressions and psycholinguistic theories of adaptation: it highlights the need for dynamic semantic representations and suggests that listeners integrate their general linguistic knowledge with speaker-specific experiences to arrive at more precise interpretations.



