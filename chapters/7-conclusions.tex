%\subsection{Methodological implications}

%My results also have implications for conducting psycholinguistic experiments. First,
%the finding that listeners adapt to the statistics of their environment within a short experiment
%suggests that experimenters should be cognizant of potential adaptation effects when probing
%production expectations or interpretations of uncertainty expressions \parencite[see also][]{Jaeger2010}. 

%Further, the results of Experiment~1, and in particular, the finding
%that participants' expectations about the use of utterances in the experiment strongly depended on
%the alternative utterances that we provided, highlights the need to be cautious about drawing general conclusions about expectations of use from single experiments. For example,
%had we only considered the results from the \textit{bare-might} condition (see \figref{fig:norming-results-main}),
%we might have concluded that ``might'' is an expected expression to communicate an event probability of 75\%,
%whereas if we had only considered the results from the \textit{might-probably} condition we might have instead concluded that it is \emph{not} an expected expression to communicate an event probability of 75\%.
%This is where explicit modeling of the sort we have engaged in here is hugely helpful: formulating a concrete linking function which models the effects of 
%alternatives allows for inferring the latent meanings of utterances by combining data from different experiments \parencite[see also][for similar approaches]{Franke2014,Peloquin2016}.

% \subsection{Limitations}

%One limitation of the present research is that the experimental paradigm is not interactive and that participants likely 
%engaged in meta-linguistic reasoning in providing production expectation and interpretation ratings. 
%While we tried to make the communicative situation depicted in the experiments natural,
%the paradigm is clearly different from everyday dialog. This limitation was necessary for the tight coupling between the experimental work
%and the model simulations that allowed us to investigate what kind of representations listeners update during adaption; in a more
%naturalistic and unconstrained setting, we would not have been able to obtain information about listener's production expectations and about their
%uncertainty in both production expectations and interpretations. However, considering that our task was different from everyday interactions, investigating
%to what extent the results in the present research translate to less scripted and more interactive settings is an important area for future research. Employing measures like eye movements or mouse-tracking could provide insight into whether participants' updated beliefs affect online language processing, i.e.~where meta-linguistic reasoning is unlikely to occur. In this vein, mouse-tracking has recently been
%employed  to investigate the incremental nature of adaptation in the domain of  prosodic cues \cite{Roettger2019}. Both eye-tracking and mouse-tracking experiments 
%allow for implementing more natural interpretation tasks while still providing information about participants' uncertain beliefs via fixation patterns or cursor trajectories.

%\epigraph{Da steh ich nun, ich armer Tor, \\
%und bin so klug als wie zuvor.} {\textit{Faust}\\ \textit{Johann Wolfgang von Goethe}}

At the beginning of this dissertation I raised the question of how listeners deal with variability in productions at the semantic-pragmatic level. The experiments that I presented in the previous chapters provided answers to several important subquestions of this larger issue: In Experiment~2, I showed that after brief exposure to a speaker, listeners rapidly update their expectations about that speaker's use of uncertainty expressions to closer match the speaker's behavior. In Experiment~3, I showed that this update in production expectations directly transfers to interpretations and consequently, listeners form speaker-specific interpretations of uncertainty expressions. In Experiments~4 and 5, I showed that listeners can adapt to multiple speakers and that they also adapt to contextual factors independent of the speaker such as the situation in which an utterance is produced. Finally, in Experiments~7 and 8, I showed that the extent of adaptation depends on prior expectations of language use and that to some extent listeners explain away otherwise unexpected behavior when presented with a cause.

Moreover, the modeling results in Chapter~4 provided novel insights into the cognitive processes responsible for adaptation. I found that a model based on Bayesian belief updating predicts both production expectation and comprehension data well. Further, I found that a model that assumes that listeners update both semantic representations and speaker preferences predicts post-adaptation behavior better than models that assume that only one of these two types of representations are updated.

What do these results tell us about the human comprehension system? As I mentioned in the introduction, there have been three proposals for dynamic comprehension systems that can deal with variability: \textit{normalization} of input, \textit{alignment} of linguistic representations, and \textit{adaptation} to variable language use. As I also discussed in Chapter~4, \textit{normalization} is not applicable in the interpretation of uncertainty expressions since listeners are faced with the challenge of mapping a discrete expression to a continuous event probability rather than mapping a continuous property to a discrete symbol as is the case with the recognition of phonemes. The other two accounts, \textit{alignment} and \textit{adaptation}, on the other hand, were both plausible candidates for the processes leading to partner-specific in the interpretation of uncertainty expressions.

The experimental evidence that I provided in this dissertation adjudicates between these two accounts. On the one hand, all my results are compatible with a sophisticated adaptation account according to which listeners constantly update speaker expectations based on statistical input. This account predicts that listeners form speaker-specific production expectations, that listeners form speaker-specific interpretations, that listeners can adapt to multiple speakers, and that listeners' adaptation behavior depends on prior production expectations, which may be affected by non-linguistic contextual factors.

An alignment account, on the other hand, is only compatible with some of the findings from the experiments above. If one assumes that linguistic representations are linked to contextual representations,  an alignment account predicts that listeners align speaker expectations and interpretations to a single speaker. However, importantly, this account fails to account for adaptation to multiple speakers. As I explained in Chapter~2, this account is based on the assumption that partner-specific behavior is a result of residual activation of linguistic representations from comprehending and producing previous utterances in interaction. Since residual activation  of representations decreases over time, one would expect partner-specific behavior to be limited to interactions with the most recent interlocutor rather than -- as I found in Experiment~4 -- listeners forming different production expectations for multiple speakers. Further, an alignment account, which assumes that partner-specific behavior is an by-product of automatic activation of linguistic representations, does not predict the modulation of adaptation by non-linguistic contextual factors that I demonstrated in Chapter~6. 

\section{Generalizability of findings}

My investigations all focus on one class of linguistic expressions, namely uncertainty expressions. This raises the question
to what extent one expects the findings presented here to generalize to other linguistic phenomena.

Throughout this dissertation, I made the assumption that uncertainty expressions have a threshold semantics. Similar semantic
representations have also been successful in predicting the use of quantifiers \egcite{Scholler2017} and gradable adjectives \egcite{Kennedy2007}. Given
the parallels in meaning representations, I expect my findings to directly transfer to these two classes of linguistic expressions, and in fact, 
the results by \textcite{Yildirim2016} provide direct evidence that listeners adapt to variable use of quantifiers, and recent work by \textcite{Xiang2020} 
provides evidence that listeners also adapt to variable use of gradable adjectives.

More abstractly, the threshold distributions can be seen as a property of the lexicon and updating beliefs about threshold distributions can be seen as
updating beliefs about the lexicon. Under this view, it seems likely that many of the results in this dissertation apply more generally to the interpretation of most or all linguistic expressions.
In interaction, listeners can update their beliefs about the speaker's lexicon and learn more precise mappings between expressions and meanings. Evidence for such behavior comes from  experiments and models simulating the formation of conceptual pacts \cite{Hawkins2017}. According to their model, interlocutors update their beliefs about the mapping of referring expressions
to referents in a repeated reference game of describing abstract tangram figures.

Importantly, however, I do not want to imply that listeners will readily update their production expectations and interpretations for any type of expressions. For example, a listener will likely not
update their beliefs about a speaker's mapping for the word \textit{dog} in response to observing the speaker use \textit{dog} to refer to cats.\footnote{They may, however, draw other inferences such as that the speaker does not know what a dog is or that the speaker is uncooperative.} To see why this is the case, recall that one important component of the adaptation model that I presented above are prior beliefs about a speaker's productions before interacting with the speaker. I expect listeners' prior beliefs about the mapping of \textit{dog} to exhibit very little variance, since language users generally seem to agree what kind of objects \textit{dog} refers to. Therefore any lexicon in which \textit{dog} refers to cats, will have extremely low prior probabilities and thus evidence of a speaker using \textit{dog} to refer to cats will have very limited effects on posterior beliefs. In cases in which there is inter-speaker variability as with uncertainty expressions or abstract tangram figures, on the other hand,  listeners' prior beliefs exhibit more uncertainty and therefore observed behavior has a stronger effect on posterior beliefs and ultimately post-adaptation behavior.

Finally, it is also noteworthy that my results highlight a lot of parallels between phonetic adaptation behavior and semantic-pragmatic adaptation behavior. For example, I found that more exposure leads to stronger adaptation, as in phonetic adaptation \parencite{Vroomen2007}; I found that,  if listeners are presented with a cause,  they explain away otherwise unexpected behavior, as in phonetic adaptation \parencite{Kraljic2008}, and similarly as for phonetic adaptation \parencite{Kleinschmidt2015}, a model based on Bayesian belief-updating predicts post-adaptation behavior. While there is no evidence for this beyond these parallels, this might suggest that listeners employ similar processes in learning speaker-specific behaviors at all linguistic levels.



\section{Limitations}

One limitation of the present research is that the experimental paradigm is not interactive and that participants may have 
engaged in meta-linguistic reasoning in providing production expectation and interpretation ratings. 
While I tried to make the communicative situation depicted in the experiments natural,
the paradigm is clearly different from everyday dialog. This limitation was necessary for the tight coupling between the experimental work
and the model simulations that allowed me to investigate what kind of representations listeners update during adaption; in a more
naturalistic and unconstrained setting, I would not have been able to obtain information about listener's production expectations and about their
uncertainty in both production expectations and interpretations. 

Further, I did not systematically investigate to what extent listeners draw higher-level inferences rather than adapting to individual expressions. For example, it 
could be that listeners infer that the speaker wants to be very encouraging when interacting with the child rather than learning something about the speaker's mapping of
uncertainty expressions to event probabilities. While it seems likely that higher-level inferences also affect interpretations, I consider it unlikely that higher-level inferences are the
sole cause for the observed behavior in my experiments. In Chapter~3, I presented results that show that the expressed preference of the child had only a very small effect on
production expectations. Further, in the control conditions in  the explaining away experiment in Chapter~6, I was able to replicate the effects without a child being the fictional
 interlocutor of the experimental speaker. Thus, while it will be important to investigate the exact contributions of higher-level inferences, the aggregate results in 
 this dissertation do not provide evidence for a strong effect of higher-level inferences on post-adaptation behavior.

Similarly, in the model comparisons in Chapter~4, I only considered two types of beliefs that may change as a result of adaptation: beliefs about the semantics
and about preferences. While the model comparisons provided strong evidence that listeners update beliefs about the semantics, the evidence for listeners also updating
preferences was weaker and a model that allowed both types of beliefs to be updated only marginally improved the model's predictive power. 
Further, it could be that these additional parameters in the model that allows updates to preferences are in fact capturing the variance of other factors such as social factors. 
Thus, more model comparisons that consider additional factors should be conducted. However, the experimental results in this dissertation and by \textcite{Yildirim2016}
also provide independent evidence for listeners tracking speaker preferences. \textcite{Yildirim2016} and I found that varying the ratio of productions with different uncertainty expressions also has a small effect on post-adaptation behavior: 
for example, if listeners are exposed to an equal number of productions with \textit{might} and \textit{probably}, their post-adaptation production expectations exhibit less of a bias towards either
of these expressions than when they are exposed to a speaker who uses one of these expressions more often than the other.  This suggests that listeners infer preferences for uncertainty expressions depending on the frequency with which a speaker uses different terms.

Moreover, while I argued that there are a lot of similarities between phonetic adaptation and semantic-pragmatic adaptation, I also have not shown directly
that these processes use shared cognitive mechanisms and it could be that despite the similarity in behavioral results phonetic and semantic-pragmatic
are two distinct phenomena at the implementational level. This also raises the question whether \textit{adaptation} is indeed the correct term
for the phenomenon that I have been discussing. Considering that I have argued that semantic-pragmatic adaptation involves learning about a speaker's
lexicon, the process may be considered more similar to word learning and would therefore potentially be described by a term such as \textit{continuous word learning}. 

Lastly, I portrayed adaptation as a long-term phenomenon that cumulatively improves communication with known interlocutors.
However, in all my experiments, the test phase immediately followed the exposure phase and it remains an open question
whether adaptation to individual speakers indeed persists over longer periods of time (as has been established for phonetic and syntactic
adaptation; e.g., \citeauthor{Xie2018}, \citeyear{Xie2018}; \citeauthor{Kroczek2017}, \citeyear{Kroczek2017}).

\section{Future directions}

One advantage of computational models such as the one that I presented in this dissertation is that they make precise quantitative predictions which can
be subsequently tested in experiments. In this final section, I will discuss several predictions that the model makes and sketch how they could be investigated experimentally.

First, the model predicts that adaptation is incremental and listeners should incrementally update their beliefs following each interaction. While my experiments showed that more exposure leads to more adaptation, I have not systematically investigated the predicted incremental nature of semantic-pragmatic adaptation. One important future direction is therefore to employ paradigms in which exposure and test trials can be combined such as a visual world eye-tracking or mouse-tracking paradigm.  In this vein, mouse-tracking has recently been
employed  to investigate the incremental nature of adaptation in the domain of  prosodic cues \cite{Roettger2019}, and a similar paradigm could be used to study semantic-pragmatic adaptation. Another advantage of using an online measure would be that eye-tracking and mouse-tracking experiments allow for 
implementing more natural interpretation tasks while still providing information about participants' uncertain beliefs via fixation patterns or cursor trajectories,
and these experiments could also provide insight into whether participants' updated beliefs affect online language processing, i.e., where meta-linguistic reasoning is unlikely to occur.

Second, as I mentioned repeatedly, the model predicts that adaptation should depend on the extent of uncertainty reflected in prior beliefs. If listeners don't exhibit uncertainty about a speaker's productions, no or very little adaptation should occur. If, however, listeners exhibit a lot of uncertainty about a speaker's productions, then the model predicts
very rapid adaptation. This relationship between uncertainty in prior beliefs and adaptation behavior holds for the uncertainty expressions that I investigated: I found in Experiment~1 that listeners exhibit uncertainty about a generic speaker's productions. However, whether this relationship holds more generally, and whether listeners adapt much less if there is less prior uncertainty remains an open question. 

Studying this relationship between prior uncertainty and adaptation could likely be done in a comparative study with different classes of expressions. For example, if listeners exhibit different levels of prior uncertainty regarding a generic speaker's use of uncertainty expression, quantifiers, and gradable adjectives, one could run exposure-test experiments for expressions from all of these classes. The model would predict that all things being equal the size of the adaptation effect should correlate with prior uncertainty.

Another important question concerns generalization from one speaker to another speaker, or from one situation to another situation. In all the experiments reported in this dissertation the speaker's identity and all other contextual factors were the same across the exposure and test phase. One important future direction is thus to also study adaptation effects when one or more aspects of the context change. 

Results from generalization experiments could also help adjudicate between the mixture model and the hierarchical model that I discussed in Chapter~5. If the situations between the exposure and test phase are considerably different (which may be challenging to achieve given that being in an experiment may be an important contextual factor), the hierarchical model predicts that we should not see speaker-specific adaptation. This behavior is predicted by the hierarchical model  because according to this model, speaker-specific parameters are tied to a situation and if the situation changes, the model predicts that speaker expectations are guided by an independent set of speaker-specific parameters that are only linked to the novel situation. The mixture model, on the other hand, would predict transfer of speaker-specific behavior from one situation to another.

Finally, in Chapter~6, I speculated that there may be limits to adaptation and that these limits are the reason for the lack of a larger adaptation effect as compared to a control condition when the observed speaker behavior was highly incongruent with the expected behavior. However, given the limited data on this issue, the conclusions on this issue remain highly speculative and a future study should investigate whether we also observe limitations on adaptive behavior in other scenarios. For example, it could be that we would observe similar limitations in the absence of non-linguistic contextual factors if we made the exposure speaker's behavior more extreme. If this is the case, it could be that individual belief updates are capped and at some point higher surprisal no longer leads to larger belief updates in order to prevent too extreme deviations from what is expected. Thus, if limited adaptation behavior can be observed in multiple scenarios, another important future direction would be to study the relationship between prior surprisal and the size of the adaptation effect.

\pagebreak

To conclude, in this dissertation I provided multiple new insights into how listeners deal with variability at the semantic-pragmatic level. My investigations have highlighted the dynamicity of the language comprehension system and the novel computation model will hopefully serve as a starting point for many additional investigations into how listeners interpret utterances in variable environments.


