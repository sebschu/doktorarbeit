%\subsection{Methodological implications}

%My results also have implications for conducting psycholinguistic experiments. First,
%the finding that listeners adapt to the statistics of their environment within a short experiment
%suggests that experimenters should be cognizant of potential adaptation effects when probing
%production expectations or interpretations of uncertainty expressions \parencite[see also][]{Jaeger2010}. 

%Further, the results of Experiment~1, and in particular, the finding
%that participants' expectations about the use of utterances in the experiment strongly depended on
%the alternative utterances that we provided, highlights the need to be cautious about drawing general conclusions about expectations of use from single experiments. For example,
%had we only considered the results from the \textit{bare-might} condition (see \figref{fig:norming-results-main}),
%we might have concluded that ``might'' is an expected expression to communicate an event probability of 75\%,
%whereas if we had only considered the results from the \textit{might-probably} condition we might have instead concluded that it is \emph{not} an expected expression to communicate an event probability of 75\%.
%This is where explicit modeling of the sort we have engaged in here is hugely helpful: formulating a concrete linking function which models the effects of 
%alternatives allows for inferring the latent meanings of utterances by combining data from different experiments \parencite[see also][for similar approaches]{Franke2014,Peloquin2016}.

% \subsection{Limitations}

%One limitation of the present research is that the experimental paradigm is not interactive and that participants likely 
%engaged in meta-linguistic reasoning in providing production expectation and interpretation ratings. 
%While we tried to make the communicative situation depicted in the experiments natural,
%the paradigm is clearly different from everyday dialog. This limitation was necessary for the tight coupling between the experimental work
%and the model simulations that allowed us to investigate what kind of representations listeners update during adaption; in a more
%naturalistic and unconstrained setting, we would not have been able to obtain information about listener's production expectations and about their
%uncertainty in both production expectations and interpretations. However, considering that our task was different from everyday interactions, investigating
%to what extent the results in the present research translate to less scripted and more interactive settings is an important area for future research. Employing measures like eye movements or mouse-tracking could provide insight into whether participants' updated beliefs affect online language processing, i.e.~where meta-linguistic reasoning is unlikely to occur. In this vein, mouse-tracking has recently been
%employed  to investigate the incremental nature of adaptation in the domain of  prosodic cues \cite{Roettger2019}. Both eye-tracking and mouse-tracking experiments 
%allow for implementing more natural interpretation tasks while still providing information about participants' uncertain beliefs via fixation patterns or cursor trajectories.
