\section{The semantics of uncertainty expressions}

While this is not a dissertation about modality, many of the utterances that I consider in my experiments contain an epistemic modal.
Since I discuss the implications of my experimental results for popular theories of the semantic of epistemic modals in Section~{XXX}, and since 
my computational model is inspired by recent accounts of epistemic modality, I provide a brief introduction to several theories of modality in this section.

Here, and throughout this dissertation, I adapt the broad notion of modality by \cite{Portner2009} and \cite{Kratzer2012Ch2}, which not only 
includes modal auxiliaries (e.g., \textit{might}, \textit{could}) but also other evidential devices such as probability operators 
(e.g., \textit{probably}) and attitude verbs (e.g., \textit{think}). At the same time, however, to limit the scope of this discussion,  
I will only cover epistemic modality, and therefore omit any discussion of deontic modals, i.e., modals to express how the 
world should be according to laws, societal norms, etc., which are frequently discussed together with epistemic modals.
I will also omit discussions of the important connections between modals and conditionals \cite{see e.g., Lewis1973?,Kratzer1978,Kratzer1979,Kratzer2012}
and discussions of the extent to which different semantic theories validate desired and undesired logical inferences \cite{see e.g., Yalcin2010}.

\subsection{Background: Possible world semantics}

Classical modal logic and most other semantic theories of modals are based on the concept of possible worlds \cite{kripke1963}.
A possible world is a world which differs in one or multiple properties from the actual world. For example, while the proposition $\phi_{brown}$
expressed by the  sentence ``I have brown hair'' is true in the actual world $w$, one possible world $w_1$ is identical in every regard to 
the actual world except that the proposition $\phi_{blond}$ encoded by ``I have blond hair'' is true and the one encoded by
``I have brown hair'' ($\phi_{brown}$) is false. 

According to  a possible world semantics, all sentences have to be evaluated relative to a possible world $w$ and
propositions $\phi$ can be represented as a set of worlds in which $\phi$ is true. If we consider the worlds $w$ and $w_1$ as described here,
the propositions expressed by the sentences ``I have brown hair'' and ``I have blond hair'' evaluate to different truth conditions, depending on the possible world.

$$\sem{\phi_{brown}}^w = 1 \mbox{ iff } w \in \phi_{brown}  = 1$$
$$\sem{\phi_{brown}}^{w_1} = 1 \mbox{ iff } w_1 \in \phi_{brown}  = 0$$
$$\sem{\phi_{blond}}^{w} = 1\mbox{ iff } w \in \phi_{blond} = 0$$
$$\sem{\phi_{blond}}^{w_1} = 1 \mbox{ iff } w_1 \in \phi_{blond} = 1$$




%* start off with why i'm talking about modals

%* what are modals

%* what are the questions relevant to modality

%* which ones of these are relevant for this thesis

%* Possible worlds

%* Modal logic

%* Kratzer

%* threshold semantics proposals

%* wallsten/budesco




%The semantics of epistemic modals\footnote{I adapt the broad notion of modality by \cite{Portner2009} and \cite{Kratzer2012}, which not only 
%includes modal auxiliaries (e.g., \textit{might}, \textit{could}) but also other evidential devices such as probability operators 
%(e.g., \textit{probably}) and attitude verbs (e.g., \textit{think}).} such as \textit{might}, \textit{could} and \textit{probably} 
%has been extensively discussed in the formal semantics literature. However, a lot of these works focus on how 
%different meaning representation affect logical inferences and how they can be used to compositionally derive the 
%meaning of sentences with modal expressions, which are less relevant debates for the enterprise in this dissertation.
%I therefore primarily give an overview of different formalisms along with a discussion about what they predict
%about the interpretation of epistemic modals when they are used to communicate probabilities of future events.

\subsubsection{Modal logic}

In classical modal logic, the truth conditions of sentences with epistemic modals depend on an accessibility relation $R$.
$R$ determines which worlds $w'$ are epistemically accessible from the actual world $w$, i.e., which worlds are epistemically consistent with
the actual world. For example, consider rolling two six-sided dice, one after another. Before you roll the first die, all worlds in which the sum of
the two dice is between 2 and 12 (all possible combinations of two dice) are epistemically accessible since they are compatible with the actual
world. Now, if you roll one of the dice and it comes up 4, only worlds in which the sum of the two dice is between 5 and 10 (all possible sums of 4 and 
a number between 1 and 6) are epistemically accessible. 

Formally, if $wRw'$ is true then $w'$ is epistemically accessible from $w$. A proposition $\phi$ embedded under an epistemic modal is then true
if either $\phi$ is  true in all epistemically accessible worlds (for necessity modals such as \textit{must}) or $\phi$ is true in at least one epistemically 
accessible world (for possibility modals such as \textit{might}).

\begin{exe}
\ex \label{ex:modall-must} $\sem{\mbox{must } \phi}^{w}  = 1 \mbox{ iff } \forall w' \in W: wRw' \rightarrow  \sem{\phi}^{w'} = 1$
\ex \label{ex:modall-might} $\sem{\mbox{might } \phi}^{w}  = 1 \mbox{ iff } \exists w' \in W: wRw' \rightarrow  \sem{\phi}^{w'} = 1$
\end{exe}

If we again use the example of rolling two dice and assume that the world $w_x$ corresponds to the sum 
of the two dice being $x$, then $wRw'$ is true iff $w' \in \{w_2, w_3, ..., w_{11}, w_{12}\}$. Therefore, for example,
\begin{align*}
\sem{&\mbox{must roll a number between 2 and 12}}^{w} =  1 \\
 & \mbox{(since \textit{roll a number between 2 and 12} is true in all epistemically accessible worlds)} \\ \\ 
 \sem{&\mbox{must roll a 7}}^{w} =  0 \\
 & \qquad \qquad \qquad \quad \mbox{ (since \textit{roll a 7} is only true in some epistemically accessible worlds)} \\ \\
 \sem{&\mbox{might roll a 7}}^{w} =  1 \\
 &  \qquad \qquad \qquad \qquad \quad \mbox{ (since \textit{roll a 7} is true in the epistemically accessible world }w_7\mbox{)}\\ \\ 
 \sem{&\mbox{might roll a 1}}^{w} =  0 \\
 &  \qquad \qquad \qquad \qquad \qquad \mbox{\  (since \textit{roll a 1} is false in all epistemically accessible worlds).}
\end{align*}

While this approach seems intuitively correct for scenarios like rolling two dice, it is very challenging to represent utterances
that convey more fine-grained meanings than mere possibility or necessity. As \cite{Lassiter2017} points out, one could extend this proposal
to modal expressions such as \textit{probably} and \textit{likely} by assuming that \textit{probably $\phi$} is true if $\phi$ is true in more 
epistemically accessible worlds than epistemically accessible worlds in which $\phi$ is false:
\begin{exe}
\ex $\sem{\mbox{probably} \phi}^{w}  = 1 $ \\ 
 \ \ \ \ \ \ \ \ $ \mbox{ iff } |\{w' \in W \mid wRw' = 1 \land \sem{\phi}^{w'} = 1\}| > |\{w' \in W \mid wRw' = 1 \land \sem{\phi}^{w'} = 0\}|$
\end{exe}
However, this proposal comes with at least two shortcomings if one wants to consider it as a 
complete theory of epistemic modals. First, one has to make the limit assumption \cite{Lewis1981}, i.e., 
one has to assume that $W$ contains a finite number of possible worlds. Second, this proposal does not provide 
a theory of interpretation for any type of graded epistemic modal expressions such as \textit{It is 60\% likely that...} or \textit{It is highly probable that...}
or modal expressions in comparative constructions such as \textit{It is twice as likely that X than Y} \cite{Lassiter2017}. Third, there
is no connection between event probabilities and the use of different modals except that this account would predict that \textit{might $\phi$} is true
when the probability of $\phi$ is greater than 0, and \textit{must $\phi$} is true if the probability of $\phi$ is 1.

%For my experiments, neither of this will be necessarily an issue; I am not dealing with graded expression 
%and the number of future events and hence also the number of possible world is limited. 
%However, two additional issues arise. First, given the previously established inter-subject variability in the interpretation 
%of epistemic modals \cite{Wallsten1986}, we need a theory that can accommodate variability. The only possibility to represent
%variability in this formalism would be to assume that either different speakers use different accessibility relations or different speakers
%use different sets of possible worlds. However, it seems unlikely that speakers who have access to the same kind of information 
%(the same visual stimuli both in \cite{Wallsten1986} and in my experiments) would rely on different accessibility relations or a different set of 
%possible worlds.

%Second, while intuitively this definition of \textit{probably} such that $\sem{\mbox{probably} \phi}^w$ is true if $\phi$ is more likely than $\lnot \phi$ 
%appears to be reasonable, there does not seem to be a similar definition for other uncertainty expression such as \textit{think} or \textit{looks like}.
%We could again assume that there are different accessibility relations associated with 

% and unless we assume
%variable accessibility relations $R$ or variable sets of possible worlds, this theory cannot represent 

% it remains unclear how this theory could predict any type of variability. Second, 



%probably $ wRw' and true / wRw' > 0.5$ 
%might $ wRw' and true / wRw' > 0$ 
%must $ wRw' and true / wRw' = 1$
%very likely $ wRw' and true / wRw' > 0.5 + \theta$
%more likely than $ wRw' and A is true / wRw' > wRw' and B is true / wRw'$

%issues: one has to be very careful how to set up the world structure --> otherwise law of probability no longer holds

%needs again limit assumption

%how is this better from just using probabilities?

\subsubsection{Double relativity of modals}

The most prominent semantic theory of epistemic modals (and all other flavors of modals) is the account by Kratzer 
\cite{Kratzer1981,1991}, later revised in \cite{Kratzer2012}. Building on \cite{Lewis1973}, she developed a unifying 
account of all modal flavors, which assumes that there are several core meanings of modals that can be expressed 
by various linguistic devices (e.g., \textit{could} and \textit{might} are both possibility modals), and that the interpretation
of a sentence with a modal depends on two \textit{conversational backgrounds}, that is, contextually specified functions 
from possible worlds to sets of propositions: 
the \textit{modal base} $f(w)$ and the \textit{ordering source} $g(w)$. 

The intuition behind the modal base $f(w)$ is that one can explicitly state which worlds the modal quantifies over using an 
``In the view of ...'' adverbial clause. For example, for epistemic modals, the modal base may be a function that returns the set of propositions
that are compatible with what is knowns in the current world, which can be explicitly expressed in a sentence with an epistemic modal through the
adverbial clause ``In the view of what is known'':

\begin{exe}
\ex In the view of what is known, it could rain tomorrow.
\end{exe}

\noindent Kratzer argues that utterances without explicit mention of a conversational background are interpreted
by contextually resolving the relevant modal base. If we ignore the ordering source for a second, this leads to the following 
definitions of $f$-necessity and $f$-possibility.

\begin{quote}
\noindent $f$-\textit{necessity}: \\
$\phi$ is a necessity with respect to a modal base $f$ iff $\phi \subseteq \cap f(w)$.

\noindent $f$-\textit{possibility}: \\
$\phi$ is a possibility with respect to a modal base $f$ iff $\cap \{\{\phi\} \cup  f(w) \} \ne \emptyset$.
\end{quote}

The semantics of sentences with necessity modals such as \textit{must} and possibility modals such as \textit{might}
can then be expressed in terms of $f$-necessity and -possibility:

\begin{exe}
\ex $\sem{\mbox{must }\phi}^{w,f} = 1 \mbox{iff $\phi$ is an $f$-necessity in $w$} $
\ex $\sem{\mbox{might } \phi}^{w,f} = 1 \mbox{iff $\phi$ is an $f$-possibility in $w$} $
\end{exe}

The semantics of these expressions is equivalent to the classical modal logic semantics presented in (\ref{ex:modall-must})  and (\ref{ex:modall-might}), 
since the accessibility relation $R$ can be defined as $wRw' = w' \in \cap f(w)$. For this reason, this account suffers from the same
issues as the classical modal logic account: it cannot be used to derive interpretations for graded modals or comparatives.

Kratzer partially resolves these issues by introducing a second conversational background, 
the ordering source $g(w)$. $g(w)$ defines a partial preorder $\le_{g(w)}$ over the set of possible worlds $W$ such that 
$$u \le_{g(w)} v \mbox{ iff } \{ p \in g(w) \mid u \in p \} \subseteq \{ p \in g(w) \mid v \in p \}.$$
That means, $v$ is at least as close to an ideal as $u$ iff all propositions in the set of ideal 
propositions $g(w)$ that are true in $u$ are also true in $v$. In the case of epistemic modals, 
the ideal as defined by $g(w)$, is usually assumed to contain propositions corresponding to a normal course of events.

Necessity and possibility then be defined as follows.
\begin{quote}
\noindent \textit{Necessity}: \\
$\phi$ is a necessity with respect to a modal base $f$ and an ordering source $g$ iff for all $u \in \cap f(w)$, there is a $v \in \cap f(w)$
such that $u \ge_{g(w)} v$ and for all $z \in \cap f(w):$ if $v \ge_{g(w)} z$, then $z \in \phi$.

\noindent \textit{Possibility}: \\
$\phi$ is a possibility with respect to a modal base $f$ and an ordering source $g$ iff $\lnot \phi$ is not a necessity with respect to $f$ and $g$.
\end{quote}
\noindent Intuitively, the definition of necessity can be seen as further restricting the modal source such that something must be true if it is true in 
all epistemically accessible worlds that come closest to the ideal defined by the ordering source. 

To illustrate how the modal base and the ordering source work together, imagine a murder case in a small town in which 
there are four suspects A, B, C, and D who all have a motive.\footnote{Example adapted from \cite{Kratzer2012Ch2}.} Further,
there was a tourist T from Iceland in town when the murder happened. Since random tourists rarely murder somebody without
a motive, the set of normal propositions $g(w)$ in this example could be $\{a, b, c, d\}$, where each proposition corresponds to
 A, B, C, and D committing the murder, respectively. 
However, the conversational background of what is known $f(w)$ is compatible with A, B, C, D, or T being the murderer,
i.e., $\cap f(w) = \cup \{a, b, c, d, t\}$. Now, it seems natural for a police officer to utter (\ref{ex:kratzer-a-murder-might})
or  (\ref{ex:kratzer-abcd-murder-must}) but unlikely for the officer to utter (\ref{ex:kratzer-t-murder-might}).

\begin{exe}
\ex \label{ex:kratzer-a-murder-might} A might have committed the murder.
\ex \label{ex:kratzer-abcd-murder-must} A or B or C or D must have committed the murder.
\ex \label{ex:kratzer-t-murder-might} T might have committed the murder.
\end{exe}

\noindent Kratzer's account makes exactly these predictions. It predicts that (\ref{ex:kratzer-a-murder-might}) is true because 
$\lnot a$ is not a necessity and therefore $a$ is a possbility; it predicts that (\ref{ex:kratzer-abcd-murder-must}) since 
$a \lor b \lor c \lor d$ is a necessity; and it  predicts that (\ref{ex:kratzer-t-murder-might}) is false since $\lnot t = a \lor b \lor c \lor d$ is a necessity
and therefore $t$ is not a possibility.

Apart from introducing this distinction between normal courses of events and theoretically possible courses of events, Kratzer's account also
provides a semantics for comparatives using the notion of comparative possibility.\footnote{This is the revised definition of comparative possibility from
in \cite{Kratzer2012Ch2} which is slightly different from the original notion of comparative possibility in \cite{Kratzer1981}.}
\begin{quote}
\noindent \textit{Comparative possibility}: \\
$\phi$ is at least as good a possibility as $\psi$ in $w$ with respect to a modal base $f$ and an ordering source $g$ iff 
$$\lnot \exists u ( u \in \cap f(w) \land u \in \phi-\psi \land \forall v (( v \in \cap f(w) \land v \in \psi-\phi) \rightarrow v <_{g(w)} u))$$
\end{quote}
\noindent	Further, $\phi$ is a better possibility than $\psi$ iff $\phi$ is at least as good a  possibility as $\psi$ and $\psi$ is 
not at least as good as possibility as $\phi$. Using this definition, \cite{Kratzer1991} defines the semantics of \textit{probably} as
\begin{exe}
\ex $\sem{\mbox{probably } \phi}^{w,f,g}  = 1 \mbox{ iff $\phi$ is a better possibility than $\lnot \phi$}.$ 
\end{exe}
\noindent Similarly, the semantics of \textit{$\phi$ is more likely than $\psi$} can be defined as 
\begin{exe}
\ex $\sem{\phi \mbox{ is more likely than } \psi}^{w,f,g}  = 1 \mbox{ iff $\phi$ is a better possibility than $\psi$}.$ 
\end{exe}	

As compared to classical modal logic, this proposal has the advantage of providing a semantics for comparative constructions and
a semantics for \textit{probably}. However, this account still does not provide a compositional account for modal expressions and 
therefore does not provide a semantics for expressions such as \textit{very likely}. Second, this account also does not make predictions
about the use of epistemic modals to communicate and infer event probabilities. \cite{Kratzer2012} briefly discusses event probabilities and shows that one
can come up with probability measures on the set of sets of possible worlds $\mathcal{P}(W)$ such that $\phi \ge_{g(w)} \psi$ implies $P(\phi) \ge P(\psi)$. 
However, her discussion does not go beyond showing that a connection between event probabilities and possible worlds is possible and 
her semantic account of modals leaves it open how speakers and listeners map modals to event probabilities.

\subsubsection{Threshold semantics: Lassiter, Swanson, }






* epistemic modals

* probability operators

* uncertainty expressions in operation research

