\section{Why adapt?}

* to know the meaning of words

* to draw the right implicatures

* go through example from cognition paper

\section{Defining the scope}

In this dissertation, I investigate the extent of semantic and pragmatic adaptation as well as
the associated cognitive processes. That is, to what extent do listeners learn speaker-specific
meanings of words; to what extent do listeners learn speaker-specific expectations of words,
and to what extent does this speaker-specific knowledge affect interpretations of utterances? 
And what are the cognitive processes that lead to this behavior?

In this enterprise, I focus on uncertainty expressions
such as \emph{might} and \emph{probably}. Thus, all findings will only directly apply to 
adaptation to variable use of uncertainty expressions. However, uncertainty expressions 
belong to the much larger class of context-sensitive expressions -- a class of expressions for 
which it is generally assumed  that their interpretation crucially depends on contextually 
specified parameters which -- as I will show in subsequent chapters --
are also tied to the speaker's identity. Given the extensive research on the parallels between
uncertainty expressions and other context-sensitive expressions such as quantifiers and
gradable adjectives \cite{LassiterBook, SchoellerFranke?}, the results in this dissertation
should therfore also apply to any other types of context-sensitive expressions, and all
the presented models could be easily extended to other classes of expressions. 

\section{Why uncertainty expressions?}

Uncertainty expressions have several properties that make them a good testing ground for studying semantic and pragmatic
adaptation. First, there is no consistent mapping between uncertainty expressions and event probabilities \cite{e.g., Clark1990,Pepper1974}, 
which suggests that listeners have to rely on additional contextual information (such as speaker identity)
if they want to infer an event probability that a speaker intended to communicate using an uncertainty expression. Second, there is considerable inter-speaker variability 
in the use of these expressions \cite{Wallsten1986} and therefore it is likely that listeners expect different speakers to use these expressions
differently. Lastly, interpreting uncertainty expressions plays an important role in many everyday situations from the banal -- 
such as talking about the weather -- to the serious -- such as communicating about health risks 
\cite{Berry2004, Lipkus2007, Politi2007} or making financial decisions \cite{Doupnik2003}. 
Thus, listeners would benefit from tracking  how a given speaker uses these expressions. 

\section{Structure of this dissertation}

main points of this dissertation:

* comprehension is an extremely flexible process 

* updated interpretations are guided by speaker expectations

* meaning of uncertainty expressions is flexible

* adaptation interacts in complex ways with other contextual factors (e.g., mood)

