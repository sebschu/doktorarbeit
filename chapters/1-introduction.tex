Speakers vary in their language use at all linguistic levels. This is most obvious at the phonetic level, since we constantly encounter speakers with different accents than our own and who consequently pronounce words differently from us. But this variation also exists at other levels. For example, at the level of syntax, different speakers show different preferences for syntactic alternations such as the frequency with which they use passive constructions \cite{Weiner1983}. Or at the word level, different speakers use words differently and--to take a stereotypical example--the comment \textit{``The movie was good.''} by an overly enthusiastic American person is generally intended to convey a worse opinion than the same utterance by a British person.

This variability is at odds with the observation that successful communication is nevertheless possible most of the time. Unless a speaker has a very strong accent which a listener had not been previously exposed to, or a speaker uses many words in a very unexpected manner, listeners tend to be able to comprehend the utterances of our interlocutors. The ease of communication with interlocutors whose language use differs from their own suggests that listeners are equipped with a very dynamic comprehension system that can easily be adjusted to novel speakers, rather than processing utterances according to their own egocentric linguistic system.

In theory there are multiple possibilities of how such a dynamic comprehension system could work. For example, it could be that listeners \textit{normalize} the linguistic input prior to interpreting utterances such that any variability is removed before interpretation \egcite{NewmanSawusch1996}. It could also be that listeners constantly \textit{fine-tune} their linguistic representations such that they match the representations of their interlocutors \egcite{Pickering2004}. However, an increasing body of research suggests that listeners \textit{adapt} to specific speakers and learn-speaker specific language models that enable comprehension of utterances by speakers who vary in their productions \cite{Norris2003,Kraljic2007,Bradlow2008, Kamide2012,Kleinschmidt2015,Fine2016,Roettger2019}. The learning is driven by the statiscal input -- by tracking speaker-speci?c statistics, listeners can estimate accurate generative models of a speaker, i.e., models predicting how a speaker would pronounce a word or what a speaker would say in different contexts, which in return can lead to more accurate comprehension processes.

As I discuss in more detail in Chapter 2, there have been many experiments and considerable modeling work investigating the properties of adaptation processes in phonetics and syntax. At higher levels such as semantics and pragmatics, however, we still know a lot less about the extent to which listeners adapt or the associated cognitive processes. In this dissertation, I therefore investigate the extent of semantic-pragmatic adaptation through multiple experiments as well as the adaptation processes through computational modeling experiments. 

\section{Why adapt?}

Before, I turn to the specific research questions and experiments, let me explain why it is beneficial for listeners to adapt. We interact with many different speakers in our daily lives -- either truly interactively in conversation or more passively when watching TV, listening to audio or video recordings, or consuming other types of media. Thus, if as listeners, we constantly adapt to all the speakers we encounter and update speaker-specific expectations, we have to keep track of considerable amounts of information. This process clearly incurs some cost, which raises the question what the benefits of semantic-pragmatic adaptation are, and whether the benefits outweigh the cost. I will not be able to answer the latter question since I can neither quantify the 
the cost associated with adaptation nor quantify the utility of adaptation. However, to answer the first question, there exist clear benefits of semantic and pragmatic adaptation, including the following.

First, at the semantic level, listeners will be able to better infer the intended speaker meaning if they know the speaker's mapping between words and world states. For example, if I know that a speaker only uses \emph{some} to refer to quantities greater than 3, I will be better able to narrow down the state of the world after hearing \emph{``I ate some of the cookies''} than I would have been able to if I had assumed that the speaker uses \emph{some} exactly the same way as I do, which for the sake of the example, let's say is to refer to quantities greater than 0. Similarly, differences in speaker and listener meaning become even more striking if my meaning of \emph{some} is narrower than the speaker's \emph{some}: If a speaker uses \emph{some} to refer to a quantity of 4 but my meaning of \emph{some} is limited to quantities greater than 5, I will infer a state of the world that is incompatible with the actual world. \todo{make figure?}.

At the pragmatic level, adaptation can further help listeners to infer the speaker's intended meaning. To see this, note that one of the key assumptions about pragmatic reasoning is that listeners reason about alternative utterances when interpreting a speaker's utterance \cite{Grice1975, Horn1984}. For example, consider the following sentence that gives rise to a scalar implicature.

\begin{exe}
\ex Sue: It might snow tomorrow.
\ex \label{ex:snow-inf} $\rightsquigarrow$  It is not certain that it will snow tomorrow.
\end{exe}

According to Gricean pragmatic theories, listeners assume that a speaker is cooperative and arrive at the inference in (\ref{ex:snow-inf}) through a counterfactual reasoning process: they reason that if Sue had wanted to communicate that it is certain that it will snow tomorrow, Sue would have uttered the more informative statement \textit{It is certain that it will snow tomorrow} (or simply the bare assertion \emph{``It will snow tomorrow''}). Assuming that Sue knew the truth regarding the more informative sentence, it must be that the more informative statement is not true, which leads the listener to conclude (\ref{ex:snow-inf}). 

Accounts of pragmatic reasoning share the implicit assumption that listeners have precise expectations
 about the speaker's language use -- specifically, which utterance alternatives were available to the 
 speaker that they didn't use -- in different situations. Listeners can only draw correct pragmatic 
 inferences if they know what a speaker would have said to communicate alternative world states. 
 In large parts these expectations are guided by the meaning of words. as I illustrated with the cookies example above. 
However, speaker expectations also depend on other factors such as the speaker's preference for
 different lexical items or the set of alternative utterances from which they choose.
 
To illustrate how different beliefs about the meaning of words and utterance preferences can lead to different interpretations, consider the interpretation 
of the uncertainty expression \textit{probably} produced by three different hypothetical speakers. For the sake of this example, 
let us assume the only three expressions that a speaker can choose from are \textit{might}, \textit{probably}, and \textit{almost certainly}.
A listener's beliefs about the three speakers' meanings and preferences are schematically illustrated in Figure~\ref{fig:inference-example}.

% Figure 1: plots/fig-1-implicatures.pdf (2-column figure)
\begin{figure}
\center
\includegraphics[width=\textwidth]{plots/fig-1-implicatures.pdf}
\caption{Lexica, utterance preferences and likely interpretation of \textit{probably} for three different hypothetical speakers. The region of the probability scale covered by each line in the Lexicon panel indicates the corresponding expression's literal semantics. Height of bars in the Cost panel indicates the speaker's cost (dispreference) for each expression.}
\label{fig:inference-example}
\end{figure}

First, consider speaker {\bf A}, for whom \textit{might} is true if the described event probability (e.g., of snowing) exceeds 10\%, 
\textit{probably} if the event probability exceeds 60\% and \textit{almost certainly}  if the event probability exceeds 90\%.  
If a listener has accurate beliefs about {\bf A}'s mapping between expressions and event probabilities and observes {\bf A} 
produce the sentence \emph{It will probably snow}, they will be likely to infer a probability of snowing between 60 and 90\%. 
As illustrated above, the reasoning follows the schema of a standard scalar implicature \cite{Grice1975, Horn1984}: if  {\bf A} 
had intended to communicate a probability above 90\%, they could have said \emph{It will almost certainly snow}, which would 
have been more informative and equally relevant. Assuming the speaker knows the actual event probability and is cooperative, 
it is therefore likely that the intended probability is not above 90\%.\footnote{Under a standard Gricean view, the negation of the 
stronger alternative is inferred categorically. However, I adopt probabilistic language here in keeping with recent results that scalar 
inferences are more aptly viewed as probabilistic inference under uncertainty \cite{Goodman2013}.} 

Now, consider speaker {\bf B}, for whom \textit{might} is true if the event probability exceeds 30\%, 
\textit{probably} if the event probability exceeds 75\% and \textit{almost certainly}  if the event probability exceeds 95\%. If a listener has
accurate beliefs about {\bf B}'s mappings, they will be likely to infer, via the same reasoning as above, a chance of snow between 75\% and 95\% when they hear {\bf B} produce the same sentence, \textit{It will probably snow}.

Finally, consider speaker {\bf C}. {\bf C} uses the same mapping between expressions and event probabilities as {\bf B}. However, {\bf C} has a strong preference against 
producing \textit{almost certainly}. If a listener has accurate beliefs about {\bf C}'s lexicon and production preferences, 
they will be likely to infer a chance of snow between 75\% and 100\% when they hear {\bf C} produce \textit{It will probably snow} since they will not
consider  \textit{almost certainly} a likely alternative. That is, the scalar inference will be blocked by the additional knowledge of the speaker's production preferences. 

As this example shows, a listener who tracks the variability in these hypothetical speakers' meanings and production preferences 
will draw on average more accurate inferences about the world than a listener who relies on their own meanings and preferences
for interpreting utterances.

The third advantage of adaptation is related to online language processing. The last several decades
in psycholinguistic research produced a lot of evidence that listeners constantly engage in the prediction of 
the upcoming input \egcite{Kuperberg2016}.\footnote{This is more generally true for many perceptual 
processes including vision \parencite[see, e.g.,][]{Clark2013,Friston2010}.} 
On the one hand, this form of predictive processing makes language comprehension more robust. If due to noise,
a listener is unable to perceive part of the signal, they can often fill in the blanks using their predictive model.
On the other hand, predictive processing makes comprehension more efficient. 
If listeners constantly predict the upcoming signal, the early stages of comprehending an utterance 
reduce to comparing predictions about the upcoming linguistic input to the perceived input and, at these stages, 
listeners only have to process this difference, i.e., the error signal. However, according to such an account,
rapid and accurate processing is only possible if listeners are able to make reliable predictions about the upcoming
input. Accurate predictions in a variable and constantly changing environment, in return, are only possible through
 adaptation which highlights another advantage of constant adaptation.

\section{Defining the scope}

In this dissertation, I investigate the extent of semantic and pragmatic adaptation as well as
the associated cognitive processes. That is, to what extent do listeners learn speaker-specific
meanings of words; to what extent do listeners learn speaker-specific expectations of words,
and to what extent does this speaker-specific knowledge affect interpretations of utterances? 
And what are the cognitive processes that lead to this behavior and what is the nature of the
representations that are updated as a result of adaptation?

In this enterprise, I focus on uncertainty expressions
such as \emph{might} and \emph{probably}. Thus, all findings will only directly apply to 
adaptation to variable use of uncertainty expressions. However, uncertainty expressions 
belong to the much larger class of context-sensitive expressions, a class of expressions for 
which it is generally assumed  that their interpretation crucially depends on contextually 
specified parameters which--as I will show in subsequent chapters--are
also tied to the speaker's identity. Given the extensive research on the parallels between
uncertainty expressions and other context-sensitive expressions such as quantifiers and
gradable adjectives \cite{Lassiter2016, Scholler2017}, the results in this dissertation
should therefore also apply to any other types of context-sensitive expressions, and all
the presented models could be easily extended to other classes of expressions. 

\section{Why uncertainty expressions?}
\label{sec:why-uncertainty-expressions}

Uncertainty expressions have several properties that make them a good testing ground for studying semantic and pragmatic
adaptation. First, there is no consistent mapping between uncertainty expressions and event probabilities \egcite{Clark1990,Pepper1974}, 
which suggests that listeners have to rely on additional contextual information (such as speaker identity)
if they want to infer an event probability that a speaker intended to communicate using an uncertainty expression. Second, there is considerable inter-speaker variability 
in the use of these expressions \cite{Wallsten1986} and therefore it is likely that listeners expect different speakers to use these expressions
differently. Lastly, interpreting uncertainty expressions plays an important role in many everyday situations from the banal -- 
such as talking about the weather -- to the serious -- such as communicating about health risks 
\cite{Berry2004, Lipkus2007, Politi2007} or making financial decisions \cite{Doupnik2003}. 
Thus, listeners would benefit from tracking  how a given speaker uses these expressions. 

\section{Structure of this dissertation}

In Chapter~2, I provide background on three topics that I repeatedly
touch upon in the main part of the dissertation: partner-specific linguistic behavior, 
the semantics of uncertainty expressions, and game-theoretic models of pragmatic reasoning.
In Chapter~3, I present experiments that investigate how English language users expect a generic speaker
to use uncertainty expressions, and I use this data to estimate the parameters of a computational model of 
production expectations of a generic speaker. In Chapter~4, I then turn to several research questions concerning
semantic-pragmatic adaptation. I first establish in experiments that listeners adapt to variable use of the uncertainty
expressions \textit{might} and \textit{probably}. I then present a computational model of the adaptation process which
is based on the generic speaker expectation model. This model allows me to run different adaptation simulations and
to investigate the nature of representations that are updated during adaptation. I further discuss the predictions 
of the model concerning the interpretation of uncertainty expressions after adaptation, and I validate these predictions
in another experiment. In Chapter~5, I show that listeners can adapt to multiple speakers who use uncertainty expressions differently,
and I discuss the implications for different models of generalization. In Chapter~6, I investigate to what extent the adaptation
process can be modulated by non-linguistic factors and show that information about a speaker's mood both influences
listeners' expectations before adaptation as well as their propensity to adapt. In Chapter~7, I discuss what the findings in this
dissertation taken together tell us about semantic-pragmatic adaptation, and I outline promising future directions.

This dissertation is primarily focused on adaptation and therefore of primary interest to researchers studying linguistic adaptation.
However, given that I also conduct many experiments probing the use of several epistemic modals, I also provide novel data points in Chapter~3
for researchers interested in epistemic modality.



